% arXiv-friendly version of the paper
\documentclass[12 pt]{amsart}

\usepackage{amsmath}
\usepackage{amssymb}
\usepackage{amsfonts}

\usepackage[largesc]{newpxtext}
\usepackage[varbb]{newpxmath}

% wider margins just for now, so that todonotes are easily visible
\usepackage[margin = 1.5 in]{geometry}
\usepackage{microtype}

% theorem setup
\usepackage{amsthm, thmtools, thm-restate}
\newtheorem{theorem}{Theorem}[section]
\newtheorem{proposition}[theorem]{Proposition}
\newtheorem{corollary}[theorem]{Corollary}
\newtheorem{lemma}[theorem]{Lemma}
\newtheorem{conjecture}[theorem]{Conjecture}
\theoremstyle{definition} % definition style
\newtheorem{definition}[theorem]{Definition}
\newtheorem{example}[theorem]{Example}
\theoremstyle{remark} % remark style
\newtheorem{remark}[theorem]{Remark}

\usepackage{hyperref, cleveref}
\hypersetup{ % make links pretty and not boxed
colorlinks,
linkcolor = blue,
urlcolor = blue,
citecolor = blue}

% add your own todostyle!
\usepackage{todonotes}
\todostyle{jayden}{color = pink}
\todostyle{jasper}{color = lime}
\todostyle{alan}{color = purple}

\title{Preference-restricted parking functions}

\author[Kappler]{Alan Kappler}
\address[A. Kappler]{Harvey Mudd College, United States}
\email{akappler@g.hmc.edu}

\author[Thadani]{Jayden Thadani}
\address[J. Thadani]{Harvey Mudd College, United States}
\email{jthadani@hmc.edu}

% \author[Bown]{Jasper Bown}
% \address[J. Bown]{Harvey Mudd College, United States}
% \email{abown@g.hmc.edu}

\begin{document}
	

\begin{abstract}
	We demonstrate a new paradigm for solving parking problems --- preference-restricted parking functions. By considering parking functions of length $n$ where all cars prefer spots in some convenient $S \subset [n]$, we convert questions about complicated parking procedures into more easily understood questions about ordinary parking functions. In particular, we demonstrate choices of $S$ that yield new combinatorial insights to the results in \cite{cameron-johannsen-prellberg-schweitzer-2008} and a proof of Abel's binomial theorem, an elementary proof of a difficult result of \cite{blake-konheim-1977} and a new characterisation of prime parking functions respectively.
\end{abstract}

\maketitle

\pagebreak

\tableofcontents

\pagebreak

% just set up to make sure the todo list gets printed fine at the start of the doc
\makeatletter
\providecommand\@dotsep{5}
\makeatother
\listoftodos\relax

\pagebreak

\section{Introduction}

\section{Initial segment restrictions}

\begin{restatable}{theorem}{resPFcount}
	\label{thm:resPFcount}
	\[
		\begin{split}
			\# \mathrm{PF}_{n \mid [s]} & = s^{n} - \sum_{i = 0}^{s - 1} \binom{n}{i} (i + 1)^{i - 1} (s - i - 1)^{n - i} \\
				    & = \sum_{i = s}^{n} \binom{n}{i} (i + 1)^{i - 1} (s - i - 1)^{n - i}.
		\end{split}
	\]
\end{restatable}

\begin{restatable}{theorem}{resPFcount1}
    \label{thm:resPFcount1}
    Let $1\le s\le n.$ Then the number of functions $\pi:[n]\to[s]$ such that $\pi$ is a parking function is
    \[s^{n} - \sum_{i = 0}^{s - 1} \binom{n}{i} (i + 1)^{i - 1} (s - i - 1)^{n - i}.\]
\end{restatable}

\begin{restatable}{theorem}{resPFcount2}
    \label{thm:resPFcount1}
    Let $1\le s\le n.$ Then the number of functions $\pi:[n]\to[s]$ such that $\pi$ is a parking function is
    \[s^{n} - \sum_{i = 0}^{s - 1} \binom{n}{i} (i + 1)^{i - 1} (s - i - 1)^{n - i}.\]
\end{restatable}

\begin{restatable}{theorem}{resPFcount2}
    \label{thm:resPFcount1}
    For $1\le s\le n,$ the number of functions $\pi:[n]\to[s]$ such that $\pi$ is a parking function may also be expressed as
    \[\sum_{i = s}^{n} \binom{n}{i} (i + 1)^{i - 1} (s - i - 1)^{n - i}.\]
\end{restatable}

\begin{theorem}
    \label{thm:resPPFcount1}
    For $1 \le s < n,$, the number of functions $\pi:[n]\to[s]$ such that $\pi$ is a prime parking function is 
    \[s^{n} - (s - 1)^{n} - \sum_{i = 1}^{s} \binom{n}{i} (i - 1)^{i - 1} (s - i)^{n - i}.\]
\end{theorem}

\begin{theorem}
    \label{thm:resPPFcount2}
    The number of functions $\pi:[n]\to[s]$ such that $\pi$ is a prime parking function is also
    \[\sum_{i = s + 1}^{n} \binom{n}{i} (i - 1)^{i - 1} (s - i)^{n - i}\]
\end{theorem}

\begin{theorem}
    \label{thm:res-1s-enumerator}
    The sum over all $\pi:[n]\to[s]$ such that $\pi$ is a parking function of $x^{|\pi^{-1}(\{1\})|}$ may be expressed as
    \[s^{n} - \sum_{i = 0}^{s - 1} \binom{n}{i} (x + i)^{i - 1} (s - i - 1)^{n - i},\]
    or as
    \[\sum_{i = s}^{n} \binom{n}{i} (x + i)^{i - 1} (s - i - 1)^{n - i}.\]
\end{theorem}

\section{Prime parking functions}

\begin{restatable}{theorem}{primeIsRes}
	There is a bijection between prime parking functions and $[n] \setminus \{ 2 \}$-restricted parking functions (of length $n$). Thus,
	\[
		\# \mathrm{PPF}_{n} = \# \mathrm{PF}_{n \mid [n] \setminus \{ 2 \}} 
	\]
\end{restatable}

\begin{restatable}{theorem}{resPrimeIsRes}
	There is a bijection between $S$-restricted prime parking functions and $T$-restricted parking functions (of length $n$) where
	\[
		T = \{ 1 \} \cup \{ i + 1 \mid i \in S, 1 < i < n \}.
	\]
	Thus,
	\[
		\# \mathrm{PPF}_{n \mid S} = \# \mathrm{PF}_{n \mid T}.
	\]
\end{restatable}

\section{Modular restrictions}

\begin{restatable}{theorem}{modPFcount}
	\label{thm:modPFcount}
	The number of parking functions of length $gs - 1$ with gap $g$ between possible preferred spots is
	\[
		\# \mathrm{PF}_{gs - 1 \mid S} = s^{gs - 2}
	\]
	where $S$ is the set of the first $s$ natural numbers $j$ with $j \equiv 1 \pmod g$.
\end{restatable}

\begin{restatable}{theorem}{modPFcount2}
	The number of parking functions of length $gs - 2$ cars with gap $g$ between possible preferred spots is
	\[
		\#PF_{gs - 2 \mid S} = s^{gs - 3} - \frac{1}{2} \sum_{i = 1}^{s - 1} \binom{gs - 2}{gs - 1} i^{gi - 2} (s - i)^{g(s - i) - 2}
	\]
	where $S$ is the set of the first $s$ natural numbers $j$ with $j \equiv 1 \pmod g$.
\end{restatable}

\bibliography{references}
\bibliographystyle{alpha}

\end{document}
