% arXiv-friendly version of the paper
\documentclass[12 pt]{amsart}

\usepackage{amsmath}
\usepackage{amssymb}
\usepackage{amsfonts}

\usepackage[largesc]{newpxtext}
\usepackage[varbb]{newpxmath}

% wider margins just for now, so that todonotes are easily visible
\usepackage[margin = 1.5 in]{geometry}
\usepackage{microtype}

% theorem setup
\usepackage{amsthm, thmtools, thm-restate}
\newtheorem{theorem}{Theorem}[section]
\newtheorem{proposition}[theorem]{Proposition}
\newtheorem{corollary}[theorem]{Corollary}
\newtheorem{lemma}[theorem]{Lemma}
\newtheorem{conjecture}[theorem]{Conjecture}
\theoremstyle{definition} % definition style
\newtheorem{definition}[theorem]{Definition}
\newtheorem{example}[theorem]{Example}
\theoremstyle{remark} % remark style
\newtheorem{remark}[theorem]{Remark}

\usepackage{hyperref, cleveref}
\hypersetup{ % make links pretty and not boxed
colorlinks,
linkcolor = blue,
urlcolor = blue,
citecolor = blue}

% add your own todostyle!
\usepackage{todonotes}
\todostyle{jayden}{color = pink}
\todostyle{jasper}{color = lime}
\todostyle{alan}{color = purple}

\title{Preference-restricted parking functions}

\author[Kappler]{Alan Kappler}
\address[A. Kappler]{Harvey Mudd College, United States}
\email{akappler@g.hmc.edu}

\author[Thadani]{Jayden Thadani}
\address[J. Thadani]{Harvey Mudd College, United States}
\email{jthadani@hmc.edu}

% \author[Bown]{Jasper Bown}
% \address[J. Bown]{Harvey Mudd College, United States}
% \email{abown@g.hmc.edu}

\begin{document}
	

\begin{abstract}
	We demonstrate a new paradigm for solving parking problems --- preference-restricted parking functions. By considering parking functions of length $n$ where all cars prefer spots in some convenient $S \subset [n]$, we convert questions about complicated parking procedures into more easily understood questions about ordinary parking functions. In particular, we demonstrate choices of $S$ that yield new combinatorial insights to the results in \cite{cameron-johannsen-prellberg-schweitzer-2008} and a proof of Abel's binomial theorem, an elementary proof of a difficult result of \cite{blake-konheim-1977} and a new characterization of prime parking functions respectively.
\end{abstract}

\maketitle

\pagebreak

\tableofcontents

\pagebreak

% just set up to make sure the todo list gets printed fine at the start of the doc
\makeatletter
\providecommand\@dotsep{5}
\makeatother
\listoftodos\relax

\pagebreak

\section{Introduction}

The classical parking function problem is to describe the parking preferences cars can have that allow every car to park under a certain parking procedure. Each of $n$ cars goes down a one-way street with $n$ parking spots, attempts to park in its preferred spot, and then each subsequent spot until it parks.

The \emph{parking functions} $\pi : [n] \to [n]$ that take cars to a list of preferences that allows everyone to park are well understood. However, many of the variants of the parking function question considered in the literature do not fully leverage our knowledge of parking functions ---
\begin{enumerate}
	\item What if there are fewer cars than spots? How can we minimize the number of cars unable to park? (A special case of the variants considered in \cite{cameron-johannsen-prellberg-schweitzer-2008}.)
	\item What if each parking spot can accommodate more than one car? (Perhaps, as in \cite{blake-konheim-1977}, our parking spots are hash buckets, and our cars are data.)
	\item Prime parking functions.
\end{enumerate}
We will show that all of these can be understood as \emph{$S$-restricted parking functions} --- parking functions $\pi : [n] \to S \subset [n]$. 

These reinterpretations yield more complete theories of the variants, new combinatorial insights \todo[jayden]{remove if we don't have room for any of the equivalence class results on $[s]$-restrictions}, simpler proofs, and connections to deep results in combinatorics. In some sense, $S$-restricted parking functions is a paradigm for solving parking questions that strikes the balance between the power of $\mathbf{u}$-parking functions and the obvious combinatorial intuition of the original parking function question.

\section{Initial segment restrictions}

\begin{restatable}{theorem}{resPFcount}
	\label{thm:resPFcount}
	\[
		\begin{split}
			\# \mathrm{PF}_{n \mid [s]} & = s^{n} - \sum_{i = 0}^{s - 1} \binom{n}{i} (i + 1)^{i - 1} (s - i - 1)^{n - i} \\
				    & = \sum_{i = s}^{n} \binom{n}{i} (i + 1)^{i - 1} (s - i - 1)^{n - i}.
		\end{split}
	\]
\end{restatable}

\begin{restatable}{theorem}{resPFcount1}
    \label{thm:resPFcount1}
    Let $1\le s\le n.$ Then the number of $[s]$-restricted parking functions is
    \[s^{n} - \sum_{i = 0}^{s - 1} \binom{n}{i} (i + 1)^{i - 1} (s - i - 1)^{n - i}.\]
\end{restatable}

\begin{restatable}{theorem}{resPFcount2}
    \label{thm:resPFcount2}
    The number of $[s]$-restricted parking functions may be expressed as
    \[\sum_{i = s}^{n} \binom{n}{i} (i + 1)^{i - 1} (s - i - 1)^{n - i}.\]
\end{restatable}

\begin{theorem}
    \label{thm:resPPFcount1}
    For $1 \le s < n,$, the number of $[s]$-restricted parking functions which are also prime is 
    \[s^{n} - (s - 1)^{n} - \sum_{i = 1}^{s} \binom{n}{i} (i - 1)^{i - 1} (s - i)^{n - i}.\]
\end{theorem}

\begin{theorem}
    \label{thm:resPPFcount2}
    For $1 \le s < n,$, the number of $[s]$-restricted parking functions which are also prime may be expressed as
    \[\sum_{i = s + 1}^{n} \binom{n}{i} (i - 1)^{i - 1} (s - i)^{n - i}\]
\end{theorem}

\subsection{Abel's binomial theorem}

In enumerating $[s]$-restricted parking functions and their prime variants, we demonstrated two different counts for each --- one by excluding all of the restricted preference lists that aren't the desired type of parking function, another by taking all parking functions and allowing those with forbidden preferences to cancel. The expressions we get from these two pairs 

Abel's binomial theorem has many equivalent statements, but perhaps most common is the following version:

\begin{restatable}[Abel's binomial theorem]{theorem}{abelsbin}
	\[
		(x + y + n)^{n} = \sum_{i = 0}^{n} \binom{n}{i} x (x + i)^{i - 1} (y + n - i)^{n - i}.
	\]
\end{restatable}



\begin{theorem}
    \label{thm:res-1s-enumerator}
    The sum over $[s]$-restricted parking functions $\pi$ of $x^{|\pi^{-1}(\{1\})|}$ may be expressed as
    \[s^{n} - \sum_{i = 0}^{s - 1} \binom{n}{i} (x + i)^{i - 1} (s - i - 1)^{n - i},\]
    or as
    \[\sum_{i = s}^{n} \binom{n}{i} (x + i)^{i - 1} (s - i - 1)^{n - i}.\]
\end{theorem}

\section{Prime parking functions}

Prime parking functions also have a characterisation in terms of $S$-restricted parking functions --- prime parking functions are in bijection with parking functions where no car prefers the second spot.

\begin{restatable}{theorem}{primeIsRes}
	There is a bijection between prime parking functions and $[n] \setminus \{ 2 \}$-restricted parking functions (of length $n$). That is,
	\[
		\# \mathrm{PPF}_{n} = \# \mathrm{PF}_{n \mid [n] \setminus \{ 2 \}} 
	\]
\end{restatable}

\begin{restatable}{theorem}{resPrimeIsRes}
	There is a bijection between $S$-restricted prime parking functions and $T$-restricted parking functions (of length $n$) where
	\[
		T = \{ 1 \} \cup \{ i + 1 \mid i \in S, 1 < i < n \}.
	\]
	That is,
	\[
		\# \mathrm{PPF}_{n \mid S} = \# \mathrm{PF}_{n \mid T}.
	\]
\end{restatable}

\section{Modular restrictions}

In \cite{blake-konheim-1977}, Blake and Konheim consider a variant of the parking problem applicable to hash buckets. \todo[jayden]{read up and add more context on this (?)} Specifically, they consider the problem of parking cars in a lot with many rows of parking spot where cars have preferences for the rows. Using machinery from complex analysis and generating functions, they obtain an enumeration of the ``parking functions'' under this modified parking procedure. \todo[alan]{I'm writing this from memory of conversations, not anything concrete. Let me know if I've written something wrong --- Jayden}

Instead of considering $s$ rows with $g$ parking spots each \todo[jayden]{except the last row (?)}, we imagine placing these rows one after each other to form a long one-way street and only permit cars to prefer the first spot in each row. This equivalent problem is a restricted parking function! Now we are able to use the circular symmetry of parking functions to enumerate these restricted parking functions, in a similar vein to Pollak's enumeration of parking functions \todo[jayden]{cite Riordan}. 

\begin{restatable}{theorem}{modPFcount}
	\label{thm:modPFcount}
	The number of parking functions of length $gs - 1$ with gap $g$ between possible preferred spots is
	\[
		\# \mathrm{PF}_{gs - 1 \mid S} = s^{gs - 2}
	\]
	where $S$ is the set of the first $s$ natural numbers $j$ with $j \equiv 1 \pmod g$.
\end{restatable}

This is really the first case of a more general result for parking functions of length $gs - k$.

\begin{restatable}{theorem}{modPFcountGen}
	The number of parking functions of length $gs - k$ with gap $g$ between possible preferred spots is $\mathrm{PF}_{gs - k \mid S}$ and satisfies the relation
	\[
		s^{gs - k} = s \# \mathrm{PF}_{gs - k \mid S} + \sum_{n = 2}^{k} \frac{s}{n} \sum_{\lambda \vdash k, \lvert \lambda \rvert = n} \sum_{\mu \vdash s, \lvert \mu \rvert = n} \binom{gs - k}{g \mu - \lambda} \prod_{i = 1}^{n} \# \mathrm{PF}_{g \mu_{i} - \lambda_{i} \mid S}
	\]
\end{restatable}

In practice, the formula above is computationally intractable for large $g, s, k$, but can be useful to show special cases of small $k$ as below.

\begin{restatable}{example}{modPFcount2}
	The number of parking functions of length $gs - 2$ cars with gap $g$ between possible preferred spots is
	\[
		\#PF_{gs - 2 \mid S} = s^{gs - 3} - \frac{1}{2} \sum_{i = 1}^{s - 1} \binom{gs - 2}{gs - 1} i^{gi - 2} (s - i)^{g(s - i) - 2}
	\]
	where $S$ is the set of the first $s$ natural numbers $j$ with $j \equiv 1 \pmod g$.
\end{restatable}

\section{Connections and conjectures}

\bibliography{references}
\bibliographystyle{alpha}

\end{document}
