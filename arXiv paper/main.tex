% arXiv-friendly version of the paper
\documentclass[12 pt]{amsart}

\usepackage{amsmath}
\usepackage{amssymb}
\usepackage{amsfonts}

\usepackage[largesc]{newpxtext}
\usepackage[varbb]{newpxmath}

% wider margins just for now, so that todonotes are easily visible
\usepackage[margin = 1.5 in]{geometry}
\usepackage{microtype}

% theorem setup
\usepackage{amsthm, thmtools, thm-restate}
\newtheorem{theorem}{Theorem}[section]
\newtheorem{proposition}[theorem]{Proposition}
\newtheorem{corollary}[theorem]{Corollary}
\newtheorem{lemma}[theorem]{Lemma}
\newtheorem{conjecture}[theorem]{Conjecture}
\theoremstyle{definition} % definition style
\newtheorem{definition}[theorem]{Definition}
\newtheorem{example}[theorem]{Example}
\theoremstyle{remark} % remark style
\newtheorem{remark}[theorem]{Remark}

\usepackage{hyperref, cleveref}
\hypersetup{ % make links pretty and not boxed
colorlinks,
linkcolor = blue,
urlcolor = blue,
citecolor = blue}

% add your own todostyle!
\usepackage{todonotes}
\todostyle{jayden}{color = pink}
\todostyle{jasper}{color = lime}
\todostyle{alan}{color = purple}

\title{Preference-restricted parking functions}

\author[Kappler]{Alan Kappler}
\address[A. Kappler]{Harvey Mudd College, United States}
\email{akappler@g.hmc.edu}

\author[Thadani]{Jayden Thadani}
\address[J. Thadani]{Harvey Mudd College, United States}
\email{jthadani@hmc.edu}

% \author[Bown]{Jasper Bown}
% \address[J. Bown]{Harvey Mudd College, United States}
% \email{abown@g.hmc.edu}

\begin{document}
	

\begin{abstract}
	We demonstrate a new paradigm for solving parking problems --- preference-restricted parking functions. By considering parking functions of length $n$ where all cars prefer spots in some convenient $S \subset [n]$, we convert questions about complicated parking procedures into more easily understood questions about ordinary parking functions. In particular, we demonstrate choices of $S$ that yield new combinatorial insights to the results in \cite{cameron-johannsen-prellberg-schweitzer-2008} and a proof of Abel's binomial theorem, an elementary proof of a difficult result of \cite{blake-konheim-1977} and a new characterisation of prime parking functions respectively.
\end{abstract}

\maketitle

\pagebreak

\tableofcontents

\pagebreak

% just set up to make sure the todo list gets printed fine at the start of the doc
\makeatletter
\providecommand\@dotsep{5}
\makeatother
\listoftodos\relax

\pagebreak

Begin list of theorems (reorderable as necessary):

\begin{restatable}{theorem}{resPFcount}
	\label{thm:resPFcount}
	\[
		\begin{split}
			\# \mathrm{PF}_{n \mid [s]} & = n^{c} - \sum_{i = 0}^{s - 1} \binom{n}{i} (i + 1)^{i - 1} (s - i - 1)^{n - i} \\
				    & = \sum_{i = s}^{n} \binom{n}{i} (i + 1)^{i - 1} (s - i - 1)^{n - i}.
		\end{split}
	\]
\end{restatable}

\begin{restatable}{theorem}{resPFcount1}
    \label{thm:resPFcount1}
    Let $1\le c$
\end{restatable}



\begin{restatable}{theorem}{modPFcount}
	\label{thm:modPFcount}
	Let $S$ be the set of the first $s$ natural numbers $j$ with $j \equiv 1 \pmod g$. Then the number of parking functions of length $gs - 1$ with gap $g$ between possible preferred spots is $\# \mathrm{PF}_{gs - 1 \mid S} = s^{g s - 2}$.
\end{restatable}

\bibliography{references}
\bibliographystyle{alpha}

\end{document}
