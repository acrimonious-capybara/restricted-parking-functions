\section{Fall week 1}

\subsection{Inclusion-exclusion proof!}

\begin{theorem}
	The number of $[s]$-restricted parking functions of length $c$ is
	\[
		\# \mathrm{PF}_{c \mid [s]} = \sum_{i = s}^{c} (-1)^{c - i} \binom{c}{i} (i + 1)^{i - 1} (i + 1 - s)^{c - i}
	\]
\end{theorem}

\begin{proof}
	
See something like chapter 5 of Aigner's \emph{A Course in Enumeration} or Stanley volume 1, chapter 2 for a precise formulation of the inclusion-exclusion I'm using.

Given a subset $I \subset [c]$ with  $\# I = i$, let $f_{=}(I)$ be the number of parking functions $\pi$ with $\pi([c] \setminus I) \subset [i + 1] \setminus [s]$ and $\pi(I) \subset [s]$. Notice that as a result $\operatorname{range} \pi \subset [i + 1]$. Clearly, $f_{=}([c])$ is just $\# \mathrm{PF}_{c \mid [s]}$.

The subset function $f_{\subseteq}$ is given by
\[
	f_{\subseteq}(I) = \sum_{T \subset I} f_{=}(T)
\]
We claim that this counts all parking functions $\pi$ with $\operatorname{range} \pi \subset [i + 1]$ and $\pi([c] \setminus I) \subset [i + 1] \setminus [s]$ --- for any such function, $T$ is the pre-image of $[s]$. Notice that $f_{\subseteq}(I) = 0$ for  $\#I < s$. Then by inclusion-exclusion (ignoring the zero terms)
\[
	f_{=}([c]) = \sum_{I \subset [c], i \ge s} (-1)^{c - i} f_\subseteq (I).
\]
Now we just need to show that $f_{\subseteq}(I)$ is $(i + 1)^{i - 1}(i + 1 - s)^{c - i}$.
\end{proof}
